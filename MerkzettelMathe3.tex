\documentclass[a4paper, 10pt, landscape]{article}

\usepackage{multicol}
\usepackage[german,ngerman]{babel}
\usepackage[utf8]{inputenc}
\usepackage[T1]{fontenc}
\usepackage{fancyhdr}
\usepackage{amsmath}
\usepackage{amssymb}
\usepackage{graphicx}
\usepackage{fancyhdr}
\usepackage{geometry}
\usepackage{paralist}
\usepackage{sfmath}
\usepackage[compact]{titlesec}
\titlespacing{\section}{0pt}{1.5ex}{0ex}
\titlespacing{\subsection}{0pt}{1ex}{0ex}
\titlespacing{\subsubsection}{0pt}{0.5ex}{0ex}
\setlength\parindent{0pt}
\geometry{
    left=0.5cm,
    right=0.5cm,
    top=0.5cm,
    bottom=0.5cm,
    bindingoffset=5mm
}
%\setlength{\columnseprule}{0.4pt}
\author{Johann Wagner, Jasper Orschulko}

%\pagestyle{fancy}

\lhead{Mathe 3 - Klausur Pott}
\rhead{\thepage}
\cfoot{Johann Wagner, Jasper Orschulko}
\rfoot{\thepage}

\renewcommand{\headrulewidth}{0.4pt}
\renewcommand{\footrulewidth}{0.4pt}
\renewcommand{\familydefault}{\sfdefault}

\newcommand{\bspVec}{\ensuremath{\begin{pmatrix} x_1\\ \vdots \\ x_n \end{pmatrix}}}
\newcommand{\limTo}[1]{ \lim\limits_{x \rightarrow #1}}
\newcommand{\limFromTo}[2]{ \lim\limits_{#2 \rightarrow #1}}

\newenvironment{alignTab}{$\begin{aligned}}{\end{aligned}$}

\title{Mathe III - Klausurhilfe}

\begin{document}
    \begin{multicols}{5}   
	\begin{small}
	\section{Allgemeines}
		\subsection{Binomische Formeln}
			\begin{align*}
				(a+b)^{2} &= a^{2}+2ab+b^{2}\\
			    (a-b)^{2} &= a^{2}-2ab+b^{2}\\
			    a^{2}-b^{2} &= (a+b)\cdot(a-b)
			\end{align*}
		\subsection{Potenzgesetze}
			\begin{multicols}{2}
	            $a^m \cdot a^n = a^{m+n}$\\
	            $a^n \cdot b^n = (ab)^n$\\
	            $\frac{a^n}{a^m} = a^{n-m}$\\
	            $\frac{a^n}{b^n} = \left(\frac{a}{b}\right)^n$\\
	            $(a^n)^m = a^{mn}$\\
	            $a^{-n} = \frac{1}{a^n}$\\
	            $log_b(1) = 0$
            \end{multicols}
         \subsection{Logarithmus-Gesetze}
		     \noindent
	         $x = log_a(y) \Leftrightarrow y = a^x$\\
	         $log(x) + log(y) = log(xy)$\\
	         $log(x) - log(y) = log(\frac{x}{y})$\\
	         $log_a(x) = \frac{log_b(x)}{log_b(a)}$  \\
	         $log(u^r) = r \cdot ln(u)$
	         \begin{multicols}{2}
		         \noindent
		         $ln(1) = 0$ \\
		         $ln(e) = 1$ \\
		         $ln(e^x) = x$ \\
		         $e^{ln(x)} = x$
	         \end{multicols}
        \subsection{Komplexe Zahlen}
	        $(a + bi) \pm (c + di) = (a \pm c) + (c \pm d)i$\\
	        $(a + bi) \cdot (c + di) = (ac - bd) + (ad + bc)i$\\
	        
            $\displaystyle \frac{a + bi}{c + di} = \frac{ac + bd}{c^2 + d^2} + \frac{cb - ad}{c^2 + d^2}i$
	\section{Integralrechnung}
	    $e^{Foo}$ u.ä. muss vorher substituiert werden!\\
        $\begin{matrix}
        \text{Funktion} & \text{Aufleitung} \\
        c & c \cdot x \\
        x^a, a \neq -1 & \frac{x^{a+1}}{a+1}\\
        x^{-1}, x \neq 0 & ln(|x|)\\
        e^x & e^x \\
        a^x & \frac{a^x}{ln(a)} \\
        sin(x) & -cos(x)\\
        cos(x) & sin(x)
        \end{matrix}
        $
        \subsection{Partielle Integration}
	        Wenn $u$ und $v$ zwei differenzierbare Funktionen sind, dann gilt: \\
	        $\int u' \cdot v = (u \cdot v) - \int u \cdot v'$
        \subsection{Substitutionsregel}
	        $\int f(g(x)) \cdot g'(x) dx = \int f(y) dy$
	        \begin{align*}
	            \int \frac{1}{5x - 7} dx &= ?\\
	            z &= 5x - 7 \\
	            \frac{dz}{dx} &= 5 \\   
	            \frac{dz}{5} &= dx  \\
	            \int \frac{1 \cdot dz}{z \cdot 5} &= \frac{1}{5} \int \frac{1}{z} dz \\
                &= \frac{1}{5} ln(z) \\
                &= \frac{1}{5} ln(5x-7)
	        \end{align*}
	\section{Ableitung}
		\subsection{typische Ableitungen}
			\begin{multicols}{2}		
				$(x)' = 1$ \\
				$(ax)' = a$ \\
				$(ax^2)' = 2ax$ \\
				$(\frac{1}{x})' = -\frac{1}{x^2}$ \\
				$(\sqrt[]{x})' = \frac{1}{2\sqrt[]{x}}$ \\
				$(ax^b)' = abx^{b-1}$ \\
				$(e^x)' = e^x $ \\
				$(a^x)' = a^x*log(a) $ \\
				$ln(x)' = \frac{1}{x}$ \\
				$(\sin x) = \cos x$ \\
				$(\cos x) = -\sin x$ \\
				$(\tan x) = \frac{1}{(\cos x)^2}$ \\ 
			\end{multicols}
		\subsection{Verknüpfungsfunktionen}
			Summenregel:\\
			$(f(x) + g(x))' = f(x)' + g(x)'$  \\ 
			Produktregel:\\
			$(f(x)g(x))' = f(x)'g(x)+g(x)'f(x)$  \\
			Quotientenregel:\\
			$(\frac{f(x)}{g(x)})' = \frac{f(x)'g(x)-g(x)'f(x)}{g(x)^2}$ \\
			Kettenregel:\\
			$(f(g(x)))' = f(g(x))'g(x)'$ \\
	\section{Stochastik}
	    $\Omega = \{ ... \}$ beschreibt den Ereignisraum und somit die Menge aller möglichen Ausgänge des Zufallsexperiments.\\
    	$A, B, C, ... \subseteq \Omega$ beschrieben ein Ereignisse des Zufallsexperimentes.\\
    	$P: \Omega \rightarrow \mathbb{R}$ ist eine Abbildung, welche jedem Ereignis eine Wahrscheinlichkeit zuordnet.\\
    	Eine Wahrscheinlichkeitsverteilung listet alle möglichen Ausgänge des Zufallsexperiments und ihre Wahrscheinlichkeiten auf.
	    \subsection{Gesetze/Axiome/...}
		    \begin{align*}
		    	P(A) &> 0 \text{ für alle } A \subset \Omega\\
		    	P(\Omega) &= 1\\
		    	P(A_1 \cap A_2) &= P(A_1) \cdot P(A_2), A_1 \cap A_2 = \emptyset \\
		    	P(A_1 \cup A_2) &= P(A_1) + P(A_2), A_1 \cap A_2 = \emptyset \\
		    	P(\Omega \backslash A) &= 1 - P(A)\\
		    	P(\emptyset) &= 0\\
		    	A \subseteq B &\iff P(A) \leq P(B) \\
		    	P(A|B) &= \frac{P(A \cap B)}{P(B)} \\
		    	&= \frac{P(B|A)\cdot P(A)}{P(B)} \\
		    	P(A\cap B) &= P(B) \cdot P(A|B)\\
		    	&= P(A) \cdot P(B|A)\\
		    	P_B(A) &= P(A|B)
	    	\end{align*}
		\subsection{Dichtefunktion}
		    $w: \mathbb{R} \rightarrow \mathbb{R}$ ist eine integrierbare, nicht negative Funktion. \\
	    	Es gilt: $\int_{-\infty}^{x} w(t) dt = F(x) = P(X \leq x)$
	    \subsection{Verteilungsfunktion}
	    	$F: \mathbb{R} \rightarrow \left[0,1\right]$ heißt Verteilungsfunktion. Verteilungsfunktion ist Aufleitung der Dichtefunktion.\\
	    	F ist rechtsseitig stetig und es gilt:
	    	\begin{align*}
	    		\limFromTo{-\infty}{x} F(x) &= 0\\
	    		\limFromTo{\infty}{x} F(x) &= 1\\
	    		P(X \geq x) &= 1 - P(X \leq x) \\
	    		&= \int_{x}^{\infty} w(t)dt\\
	    		P(a \le X \le b) &= P(X \le b) - P(X \le a)\\
	    		&= F(b) - F(a) \\
	    		&= \int_{a}^{b} w(t)dt
	    	\end{align*}
	    \subsection{Formeln}
		    E = Erwartungswert, V = Varianz
		    \begin{align*}
		    	E(X) &= \sum_{x \in X(\Omega)} x \cdot P(X = x)\\
		    	E(X) &= \int_{-\infty}^{\infty} x \cdot w(x) dx\\
		    	V(X) &= \sum_{x \in X(\Omega)} (x - E(X))^2 \cdot P(X = x)\\
			    &= \left(\sum_{x \in X(\Omega)} x^2 \cdot P(X = x)\right) - E(X)^2\\
		    	V(X) &= \int_{-\infty}^{\infty} (x - E(X))^2 \cdot w(x) dx\\
		    	&= \left(\int_{-\infty}^{\infty} x^2 w(x) dx \right) - E(X)^2	 
		    \end{align*}
		    \textbf{p-Quantile:}\\
		    Sortieren, $n\cdot p$, Einsetzen \& Index suchen, Formel anwenden:\\
			$\widetilde{X}_p=
			\begin{cases}
			\frac{1}{2}(x_{np}+x_{np+1}) & \text{falls } n \text{ ganzz.}\\
			x_{\lceil{np}\rceil} & \text{falls } n \text{ nicht ganzz.}
			\end{cases}$
	    \subsection{Verschiedene Verteilungen}
			\subsubsection{Gleichverteilung}	
				Die Gleichverteilung ist die einfachste Verteilung. Jede Möglichkeit hat die gleiche Wahrscheinlichkeit. Ein Würfel ist gleichverteilt mit $P(x_i) = \frac{1}{6}$.\\
				\begin{align*}
					P(X = x_i) = \frac{1}{N}
				\end{align*}
				Dabei ist $N = |\Omega|$ und X eine Zufallsvariable, welche gleichverteilt ist.
			\subsubsection{Binominialverteilung}
			    Ein \textbf{Bernoulli-Experiment} ist ein Experiment, welches nur \textbf{zwei} mögliche Ausgänge $A$ und $B$ hat. Eine \textbf{Binominialverteilung} ist eine Aneinanderreihung von Bernoulli-Experimenten. Dabei \textbf{muss} der Ereignisraum \textbf{unabhängig} sein. Ein Experiment kann beliebig oft, n-Mal, wiederholt werden.
				\begin{align*}
					X = B(n, p)\\
					\Omega = \{A, B\}^n\\
					P(A) = p\\
					P(B) = 1 - p = q
				\end{align*}
	    		Es ist ein \textbf{LaPlace}-Experiment, wenn $p = q$ gilt.
			    \begin{align*}
				    P(X = k) &= {n \choose k} \cdot p^k \cdot (1-p)^{n-k}\\
				    {n \choose k} &= \frac{n!}{k! (n-k)!}
			    \end{align*}
			\subsubsection{Hypergeometrische Verteilung}
				N = Grundmenge, n = Stichprobe, k = gewünscht, M = gewünschte Eigenschaft\\
					$P(X = k) = \frac{{M \choose k} \cdot {N - M \choose n - k}}{{N \choose n}}$
		    \subsubsection{Poisson-Verteilung}
			    Die Poisson-Verteilung eignet sich für seltene Ereignisse in einem fest definierten Zeitraum.
			    \begin{align*}
				    X = P(\lambda)\\
				    \Omega = \{x \in \mathbb{R} | x \geq 0\}\\
				    P(X = k) = \frac{\lambda^k \cdot e^{-\lambda}}{k!}	
			    \end{align*}
			    Die Poisson-Verteilung kann, wenn $n \ge 50$ und $p \leq 0.1$, eine Binominialverteilung annähren.
			    \begin{align*}
				    X = B(n, p) \\
				    \lambda = n \cdot p \\\\
				    P(X = k) \sim \frac{\lambda^k \cdot e^{-\lambda}}{k!}	
			    \end{align*}
	    	\subsection{Normalverteilung}
		    	$N(\mu, \sigma^2)$ ist eine Normalverteilung. Für $\mu = 1$ und $\sigma = 1$ ist es eine Standardnormalverteilung.
		    	\begin{align*}
		    		w(x) &= \frac{1}{\sigma \sqrt{2 \pi}} e^{-\frac{1}{2}(\frac{x-\mu}{\sigma})^2}\\
		    		P(a\le x\le b)&= \Phi(\frac{b-\mu}{\sigma})-\Phi(\frac{a-\mu}{\sigma})
		    	\end{align*}
		    	Für $\Phi$ siehe Standardnormalverteilungstabelle.\\
		    	Wenn $\Phi(-x)$, dann $1-\Phi(x)$\\
	    	
		    	Wenn gilt, dass $X = N(\mu, \sigma^2)$ und $Z = N(0,1)$, dann folgt $\frac{X - \mu}{\sigma}$. \\
		    	$X_B$ ist binominalverteilt. Wenn $np(1-p)\ge9$, dann $F_B(x) \sim \Phi\left(\frac{x + 0.5 - np}{\sqrt{np(1-p)}}\right)$. \\
		    	$X_P$ ist possionverteilt. Wenn $\lambda \ge 9$, dann $F_P(x) \sim \Phi\left(\frac{x + 0.5 - \lambda}{\sqrt{\lambda}}\right)$.
	    	
	    	\subsection{Tabelle Erwartungswert/Varianz}
		    	\begin{tabular}{l|c|c}
		    		&  $E(x)$ & $V(x)$\\ \hline
		    		$B(n,p)$ & $n\cdot p$ & $n\cdot p(1-p)$\\\hline
		    		$H(n,M,N)$ & $n\cdot\frac{M}{N}$ & $n\cdot\frac{M}{N}(1-\frac{M}{N})\frac{N-n}{N-1}$\\\hline
		    		$P(\lambda)$ & $\lambda$ & $\lambda$\\\hline
		    		$N(x)$ & $\mu$ & $\sigma^2$
		    	\end{tabular}
	    	\subsection{Konfidenzintervall}
		    	$Vertrauensgrad=1-\alpha$
		    	\subsubsection{Normalverteilung}
				    $[\frac{k}{n}-z_{(1-\frac{\alpha}{2})}\frac{\sigma}{\sqrt{n}};\frac{k}{n}+z_{(1-\frac{\alpha}{2})}\frac{\sigma}{\sqrt{n}}]$\\
				    z Werte in Normalverteilungstabelle nachschlagen. 
		    	\subsubsection{T-Verteilung}
			    	Keine Varianz gegeben. Stichprobe muss vorhanden sein.\\
				    $\bar{x}=\text{arithmetisches Mittel}=\frac{\sum x}{n}$\\
				    $\sigma=\sqrt{\frac{\sum(x-\bar{x})^{2}}{n-1}}$\\
				    $[\bar{x}-t_{(1-\frac{\alpha}{2};n-1)}\frac{\sigma}{\sqrt{n}};\bar{x}+t_{(1-\frac{\alpha}{2};n-1)}\frac{\sigma}{\sqrt{n}}]$\\
				    T Werte in T-Verteilungstabelle nachschlagen.
	\section{Numerik}
		\subsection{Lagrange'sches Interpolationspolynom}
			\begin{align*}
				n &= \text{Anzahl der Stützstellen}\\
				p(x) &= \sum_{i=0}^{n-1} y_i \cdot L_i(x) \\
				L_i(x) &= \prod_{j = 0, j \neq i}^{n-1} \frac{x - x_j}{x_i - x_j}
			\end{align*}
		\subsection{Newton'sches Interpolationspolynom}
			\begin{align*}
				n &= \text{Anzahl der Stützstellen}\\
				p(x) &= a_0 + a_1(x - x_0) + a_2(x - x_0)(x - x_1)\\
			    &+ a_n(x-x_0)(x - x_1)\cdot ... \cdot (x - x_n)
			\end{align*}
			Auflösen nach $a$ für die einzelnen Faktoren:
			\begin{align*}
				y_0 &= a_0\\
				y_1 &= a_0 + a_1(x_1 - x_0)	\\
				y_2 &= a_0 + a_1(x_2 - x_0) + a_2(x_2 - x_0)(x_2 - x_1)
			\end{align*}
			\subsubsection{Newton-Verfahren für Nullstellen}
				Voraussetzung: Muss stetig sein (hinschreiben!)\\
				stetig = an jeder Stelle definiert\\
				Allgemeine Formel: $x_{n+1}=x_{n}-\frac{f(x_{n})}{f'(x_{n})}$ 
	    \subsection{Newton-Cotes-Formeln}
   			a = untere Grenze\\
			b = obere Grenze\\
			$\alpha_{i,n}$ Tabelle:\\
			\begin{tabular}{l | c c c c }
				\noindent
				n & $i=0$ & $i=1$ & $i=2$ & $i=3$ \\\hline
				1 & $1/2$ & $1/2$ & & \\
				2 & $1/3$ & $4/3$ & $1/3$ & \\
				3 & $3/8$ & $9/8$ & $9/8$ & $3/8$
			\end{tabular}
			\begin{align*}
		        h &= \frac{b-a}{n}\\
		        x_i &= a+i\cdot h\\
		        p_{n}(x) &= h\cdot \sum_{i=0}^{n}\alpha_{i,n}\cdot f(x_{i})
			\end{align*}
		\subsection{Sekanten-Verfahren}
			Nur bei stetigem Intervall bestimmen\\
			\begin{compactitem}
				\item[1.] Startwerte bestimmen: $x_0$ und $x_1$
				\item[2.] $x_{n+1}=x_{n}-\frac{x_{n}-x_{n-1}}{f(x_{n})-f(x_{n-1})}\cdot f(x_{n})$
			\end{compactitem}
		\subsection{QR-Zerlegung}
			Seien $A \in \mathbb{R}^{mxn}$ mit $m \ge n$ und $rg(A) = n$.\\
			Es seien $a_1, a_2, ..., a_n \in \mathbb{R}^m$ die Spaltenvektoren von $A$. \\
			Die Vektoren $u_1, u_2, ..., u_n \in \mathbb{R}^m$ sind die Gram-Schmidt orthogonalisierten Vektoren. 
			\begin{align*}
				u_1 &= \frac{1}{|a_1|} a_1\\
				u_i' &= a_i - \sum_{j = 1}^{i-1} <u_j, a_i> \cdot u_j\\
				u_i &= \frac{u_i'}{|u_i'|}\\\\	 
				Q &= (u_1, u_2, ..., u_n)\\
				Q^{-1}\cdot A &= R 
			\end{align*}
		\subsection{LU-Zerlegung}
			%Fancy shit: https://www.youtube.com/watch?v=Gn2euB6dBQA
			L Matrizen sind Einheitsmatrizen plus:
				\begin{compactitem}
				\item[Step 1:] L1 Matrix aufbauen:\\
				$x \in \{1,2\}$\\
				$L_{x,1}=-\frac{A(x,1)}{A(1,1)}$ 
				\item[Step 2:] $\tilde{A}=L1\cdot A$
				\item[Step 3:] L2 Matrix aufbauen:\\
				$L_{3,2}=-\frac{\tilde{A}(3,2)}{\tilde{A}(2,2)}$
				\item[Step 4:] $U=L2\cdot\tilde{A}$
				\item[Step 5:] $L=L_1^{-1}\cdot L_2^{-1}$ (=Vorzeichen außerhalb Diagonale ändern.) 	
				\end{compactitem}
			\subsubsection{Lösung von PLUx = b}
				Wir berechnen zunächst ein y, welches ein Zwischenergebnis ist. Die Schritte sind sehr einfach, da L und U Dreiecksmatrizen sind.
				\begin{align*}
					P  &= Einheitsmatrix\\
					&\text{Lineares Gleichungssystem:}\\
					Ly &= P^Tb \text{ mit } P^T = P^{-1}\\
					Ux &= y
				\end{align*}
		\subsection{Jacobi-Verfahren}
			Voraussetzungen: (Schwach) Diagonaldominant und Diagonalelemente nicht null.
			Gegeben ist ein lineares Gleichungssystem mit $n$ Variablen und $n$ Gleichungen.
			$
			\begin{matrix}
				a_{11}\cdot x_1+\dotsb+a_{1n}\cdot x_n&=&b_1\\
				a_{21}\cdot x_1+\dotsb+a_{2n}\cdot x_n&=&b_2\\
				&\vdots&\\
				a_{n1}\cdot x_1+\dotsb+a_{nn}\cdot x_n&=&b_n\\
			\end{matrix}
			$
	
			Um dieses zu lösen, wird die $i$-te Gleichung nach der $i$-ten Variablen $x_i$ aufgelöst,\\
			$x_i^{(m+1)}:=\frac1{a_{ii}}\left(b_i-\sum_{j\not=i} a_{ij}\cdot x_j^{(m)}\right), \, i=1,\dotsc,n$\\
			und diese Ersetzung, ausgehend von einem Startvektor $x^{(0)}$, iterativ wiederholt.
		\subsection{Cholesky-Zerlegung}
			Voraussetzung: symmetrische Matrix \& Determinante jeder Teilmatrix > 0\\
			$A=GG^{T}$\\
			
			\begin{tiny}
				$A=\begin{pmatrix}
					g_{11}^{2} & g_{11}g_{21} & g_{11}g_{31}\\
					g_{11}g_{21} & g_{21}^{2}+g_{22}^{2} & g_{21}g_{31}+g_{22}g_{32}\\
					g_{11}g_{31} & g_{21}g_{31}+g_{22}g_{32} & g_{31}^{2}+g_{32}^{2}+g_{33}^{2}
				\end{pmatrix}$
				\begin{multicols}{2}
					$G=\begin{pmatrix}
						g_{11} & 0 & 0\\
						g_{21} & g_{22} & 0\\
						g_{31} & g_{32} & g_{33}
					\end{pmatrix}$
					$G^{T}=\begin{pmatrix}
						g_{11} & g_{21} & g_{31}\\
						0 & g_{22} & g_{32}\\
						0 & 0 & g_{33}
					\end{pmatrix}$
				\end{multicols}
			\end{tiny}
		\subsection{Matrixnormen}
			\begin{tiny}
				\begin{align*}
					\left|\bspVec\right| &= \sqrt{x_1^2 + ... + x_n^2}\\
					&... 
				\end{align*}
			\end{tiny}
	\section{Differentialgleichung}
		\subsection{DGL 1. Ordnung}	
			\subsubsection{Variation der Konstanten}
				\begin{itemize}
					\item Alle Ableitungen $y'$ umformen: \\
					$y' = \frac{dy}{dx}$
					\item Umstellen durch Integration und $e^{ln(x)}$-Trick nach $y$
				\end{itemize}
			\subsection{Anfangswertproblem}
				Wir haben unsere aufgelöste DGL: $y = C_1 \cdot ...$
				Beim AWP haben wir eine Zusatzbedingung, die ähnlich zu $y(0) = 2$ ist. AWP löst sich, indem wir einsetzen und zur Konstante umformen.
		\subsection{DGL 2. Ordnung}	
			Eine DGL kann eine Störfunktion enthalten. Störfunktionen sind für den inhomogenen Teil der Lösung verantwortlich. Jeder Teil, welcher nicht abhängig von $y^{(n)}$ ist, ist eine Störfunktion. $y(t) = y_h(t) + y_p(t)$
			\subsubsection{Charakteristisches Polynom}
				Umformen der Ableitungen: $y^{(n)} = \lambda^n$
				Anschließend werden die Lösungen für $\lambda$ bestimmt.
				
				Einfache Nullstelle:\\
				$e^{\lambda \cdot x}$\\
				k-fache Nullstelle:\\
				$x^{k-1} e^{\lambda x}$\\
				Komplexe Nullstelle:\\
				$(a \pm bi) \rightarrow e^{ax} \cdot sin(b), e^{ax} \cdot cos(b)$
				
				Bsp.: $y_h(t) = C_1 \cdot e^{2x} + C_2 \cdot e^{4x}$\\
				Bei inhomogenen DGL muss ein Ansatz gefunden werden, der zur Lösung führt, wenn man ihn samt Ableitungen in die ursprüngliche DGL einsetzt.\\
				\\
				1. Aufstellen des Ansatzes für $y = \{ \text{Ansatz} \}$\\
				2. Ableiten und Einsetzen als homogenen Teil der DGL.\\
				3. Parameter des Ansatzes ausrechnen und als $y_p$ angeben.
\end{small}
\end{multicols}
\begin{multicols}{2}
\begin{tiny}
	\section{Sin-Cos-Tan Tabelle}
		\begin{tabular}{l | c  c  c  c  c  c  c  c  c  c }
			\noindent
			$x$ & $0$ & $\frac{1}{6}\pi$ & $\frac{1}{4}\pi$ & $\frac{1}{3}\pi$ & $\frac{1}{2}\pi$ & $\frac{2}{3}\pi$ & $\frac{3}{4}\pi$ & $\frac{5}{6}\pi$ & $\pi$ & $\frac{7}{6}\pi$\\\hline
			$Grad$ & $0$ & $30$ & $45$ & $60$ & $90$ & $120$ & $135$ & $150$ & $180$ & $210$\\\hline
			$\sin$ & $0$ & $\frac{1}{2}$ & $\frac{\sqrt{2}}{2}$ & $\frac{\sqrt{3}}{2}$ & $1$ & $\frac{\sqrt{3}}{2}$ & $\frac{\sqrt{2}}{2}$ & $\frac{1}{2}$ & $0$ & $-\frac{1}{2}$\\\hline
			$\cos$ & $1$ & $\frac{\sqrt{3}}{2}$ & $\frac{\sqrt{2}}{2}$ & $\frac{1}{2}$ & $0$ & $-\frac{1}{2}$ & $-\frac{\sqrt{2}}{2}$ & $-\frac{\sqrt{3}}{2}$ & $-1$ & $-\frac{\sqrt{3}}{2}$\\\hline
			$\tan$ & $0$ & $\frac{\sqrt{3}}{3}$ & $1$ & $\sqrt{3}$ &$\pm\infty$ & $-\sqrt{3}$ & $-1$ & $-\frac{\sqrt{3}}{3}$ & $0$ & $\frac{\sqrt{3}}{3}$
		\end{tabular}
	\end{tiny}
\end{multicols}
\end{document}
