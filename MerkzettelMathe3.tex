\documentclass[a4paper,landscape, 11pt]{article}

\usepackage{multicol}

\usepackage[english]{babel}
\usepackage[utf8]{inputenc}
\usepackage{amsmath}
\usepackage{amssymb}
\usepackage{graphicx}
\usepackage{fancyhdr}
\usepackage{geometry}
\geometry{
    left=0.5cm,
    right=0.5cm,
    top=1cm,
    bottom=1cm,
    bindingoffset=5mm
}

\author{Johann Wagner, Kai Seidensticker}


\newcommand{\limTo}[1]{ \lim\limits_{x \rightarrow #1}}
\newcommand{\limFromTo}[2]{ \lim\limits_{#2 \rightarrow #1}}

\newenvironment{alignTab}{$\begin{aligned}}{\end{aligned}$}

\title{Mathe III - Klausurhilfe}

\begin{document}
    \begin{multicols}{4}
        
    
    \begin{small}
    \section{Allgemeines}
        \subsection{Potenzgesetze}
            \begin{multicols}{2}
            $a^m \cdot a^n = a^{m+n}$\\
            $a^n \cdot b^n = (ab)^n$\\
            $\frac{a^n}{a^m} = a^{n-m}$\\
            $\frac{a^n}{b^n} = \left(\frac{a}{b}\right)^n$\\
            $(a^n)^m = a^{mn}$\\
            $a^{-n} = \frac{1}{a^n}$\\
            $log_b(1) = 0$
            \end{multicols}
         \subsection{Logarithmus-Gesetze}
            \noindent
            $x = log_a(y) \Leftrightarrow y = a^x$\\
            $log(x) + log(y) = log(xy)$\\
            $log(x) - log(y) = log(\frac{x}{y})$\\
            $log_a(x) = \frac{log_b(x)}{log_b(a)}$  \\
            $log(u^r) = r \cdot ln(u)$
            \begin{multicols}{2}
            \noindent
            $ln(1) = 0$ \\
            $ln(e) = 1$ \\
            $ln(e^x) = x$ \\
            $e^{ln(x)} = x$
            \end{multicols}
        \subsection{Komplexe Zahlen}
        $(a + bi) \pm (c + di) = (a \pm c) + (c \pm d)i$\\
        $(a + bi) \cdot (c + di) = (ac - bd) + (ad + bc)i$\\
    
        $\displaystyle \frac{a + bi}{c + di} = \frac{ac + bd}{c^2 + d^2} + \frac{cb - ad}{c^2 + d^2}i$
    
         
    \section{Integralrechnung}
        $e^{Foo}$ u.ä. muss vorher substituiert werden!\\
        
        $
        \begin{matrix}
        \text{Funktion} & \text{Aufleitung} \\
        c & c \cdot x \\
        x^a, a \neq -1 & \frac{x^{a+1}}{a+1}\\
        x^{-1}, x \neq 0 & ln(|x|)\\
        e^x & e^x \\
        a^x & \frac{a^x}{ln(a)} \\
        sin(x) & -cos(x)\\
        cos(x) & sin(x)
        \end{matrix}
        $
        
        
        \subsection{Partielle Integration}
        Wenn $u$ und $v$ zwei differenzierbare Funktionen sind, dann gilt: \\
        $\int u' * v = (u * v) - \int u * v'$
        \subsection{Substitutionsregel}
        $\int f(g(x)) * g'(x) dx = \int f(y) dy$
        \begin{align}
            \int \frac{1}{5x - 7} dx &= ?\\
            z &= 5x - 7 \\
            \frac{dz}{dx} &= 5 \\   
            \frac{dz}{5} &= dx  \\
            \int \frac{1 * dz}{z * 5} &= \frac{1}{5} \int \frac{1}{z} dz \\
                                      &= \frac{1}{5} ln(z) \\
                                      &= \frac{1}{5} ln(5x-7)
        \end{align}
	\section{Numerik}
		\subsection{Lagrange'sches Interpolationspolynom}
			\begin{align*}
			n &= \text{Anzahl der Stützstellen}\\
			p(x) &= \sum_{i=0}^{n} y_i \cdot L_i(x) \\
			L_i(x) &= \prod_{j = 0, j \neq i}^{n} \frac{x - x_j}{x_i - x_j}
			\end{align*}
\end{small}
\end{multicols}
\end{document}
